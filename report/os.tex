\section{Operational Semantics}
\label{sec:os}
The operational semantics of a programming language is used to interpret a valid program into sequences of computational steps. And these sequences are the meaning of the program. In this section, I will describe the operational semantics for the generator feature.

\subsection{Sequence}
The \textit{seq} term has similar structure to the list which is constructed by taking two terms.
In my design, a \textit{yield} term would trigger a \textit{exit} pair in the reduction of \textit{seq} in order to iterate each item in a generator.
To achieve that, I defined \textit{seq_tm} for checking the \textit{seq} term and \textit{yield_tm} for checking the \textit{yield} term.
The \textit{seq_app} function is used for concatenating two \textit{seq} terms into one-dimensional \textit{seq} term since the idea of the \textit{exit} pair only works for unnested \textit{seq}.
\begin{itemize}
	\item \textbf{ST_Seq1}: If $t_1$ is a \textit{seq} term, then append $t_1$ to $t_2$. 
	\item \textbf{ST_Seq2}: If $t_1$ is not a \textit{seq} term and it is reducible, then reduce $t_1$ to $t_1$'. 
	\item \textbf{ST_Seq3}: If $v_1$ is a value and a \textit{yield} term, then return \textit{exit} pair.
	\item \textbf{ST_Seq4}: If $v_1$ is a value and a non-\textit{yield} term, then evaluate $t_2$.
\end{itemize}
\begin{lstlisting}
Reduction:
									    		seq_tm t1
								---------------------------------  (ST_Seq1)
							seq t1 t2 / st --> seq_app t1 t2 / st
								

												not (seq_tm t1)
									  	t1 / st --> t1' / st'
								--------------------------------   (ST_Seq2)
							seq t1 t2 / st --> seq t1' t2 / st'
								
													yield_tm v1
								--------------------------------   (ST_Seq3)
						seq v1 t2 / st --> pair v1 (\_:Unit t2) / st
								
												not (yield_tm v1)
								--------------------------------   (ST_Seq4)
									seq v1 t2 / st --> t2 / st
\end{lstlisting}

\subsection{Reference}
\begin{itemize}
	\item \textbf{ST_RefValue}: If $v$ is a value, then return the length is the state list as its location reference and append $v$ to the list.
	\item \textbf{ST_DerefLoc}: If $t$ is a value ($loc$ $L$) and $L$ is less than the length of the state list, then find the value under the reference cell in the list.
	\item \textbf{ST_Assign}: If $L$ is less than the length of the state list, then replace the value under reference cell (loc $L$) with $v_2$.
\end{itemize}
\begin{lstlisting}
Reduction:

								---------------------------------  (ST_RefValue)
								ref v / st --> loc |st| / st,v1

												t / st --> t' / st'
								---------------------------------  (ST_Ref)
								    ref t / st --> ref t' / st'
								
												t / st --> t' / st
								---------------------------------  (ST_Deref)
											!t / st --> !t' / st'
								
														L < |st|
								---------------------------------  (ST_DerefLoc)
								!(loc L) / st --> lookup L st / st
								
											t1 / st --> t1' / st
								---------------------------------  (ST_Assign1)
								t1 := t2 / st --> t1' := t2  / st
								
											t2 / st --> t2' / st
								---------------------------------  (ST_Assign2)
								t1 := t2 / st --> t1 := t2'  / st
								
													L < |st|
								---------------------------------  (ST_Assign)
							loc L := v2 / st --> unit / [L:=v2]st

\end{lstlisting}

\subsection{Let}
\textbf{ST_LetValue}: If $v_1$ is a value, then substitute $x$ with $v_1$ into $t_2$.
\begin{lstlisting}
Reduction:

											t1 / st --> t1' / st'
							----------------------------------  (ST_Let1)
							let x=t1 in t2 / st 
										--> let x=t1' in t2 / st'
						
							----------------------------------  (ST_LetValue)
							let x=v1 in t2  / st 
										--> [x:=v1]t2  / st'
\end{lstlisting}

\subsection{Fix}
Lambda calculus is a universal model of computation which is well defined in Chapter \textit{Stlc}. Then, we can define a general recursion with this. Let us define the \textit{fixed point} $Y = \lambda f. (\lambda g. f(g\text{ }g))(\lambda g. f(g\text{ }g))$. Then the recursion function $f$ could be defined as:
\begin{align*}
Yf &=  \lambda f. ((\lambda g. f(g\text{ }g))(\lambda g. f(g\text{ }g))) f \\
&=  (\lambda g. f(g\text{ }g))(\lambda g. f(g\text{ }g)) \\
&= f ((\lambda g. f(g\text{ }g))(\lambda g. f(g\text{ }g))) \\
&= f (\lambda f. ((\lambda g. f(g\text{ }g))(\lambda g. f(g\text{ }g))) f ) \\
&= f(Yf) = f(f(Yf)) = ...
\end{align*}
\begin{lstlisting}
Reduction:

									t1  / st--> t1' / st'
				-------------------------------------------  (ST_Fix1)
							fix t1  / st--> fix t1' / st'

				
				-------------------------------------------  (ST_FixAbs)
					fix (\xf:T1.t2) / st --> 
									[xf:=fix (\xf:T1.t2)] t2  / st
\end{lstlisting}

\subsection{While}
Since \while loop can be implemented in recursion as shown below, then we can implement \while function either in induction or lambda calculus. 
Here, $P$ is a predicate function of $x$ and $B$ is a body function of $x$ in the following function, where $x$ is reference type.
\begin{lstlisting}
def while (P,B,x) {
		if P(x) do
				B(x); while(P,B,x)
}
\end{lstlisting}
Then, the \while term can be formalized as:
\begin{lstlisting}
while (\x:(Ref T). p) (\x:(Ref T). b) =
fix \f:(Ref T)->Unit. 
		\x:(Ref T). 
		if (\x:(Ref T). p x) 
		do (seq_app (\x:(Ref T). b x) (f x)) else unit
\end{lstlisting}
In this way, (while P B x) always return a \textit{seq} term or a unit term. 
So, a complete \textit{exit} pair would be triggered by a \textit{yield} term in the reduction of the \textit{seq} term, even if a sequence is nested with a \while term that is also nested by a \textit{yield} term. In addition, I force the $f$ abstraction has a ($Ref$ $T$) $\rightarrow Unit$ type, since my \while function only takes a reference as a input and returns a $Unit$ type.
\begin{lstlisting}
Reduction:

										t1 / st --> t1' / st'
						-------------------------------------- (ST_While1)
						while t1 t2 / st --> while t1' t2 / st'
								
										t2 / st --> t2' / st'
						-------------------------------------- (ST_While2)
						while v1 t2 / st --> while v1 t2' / st'
								
						-------------------------------------- (ST_WhileFix)
					while (\x1:(Ref T). p) (\x2:(Ref T). b) / st 
					--> tfix (\f:(Ref T)->Unit. 
								   (\x:(Ref T). 
								   (if \x1:(Ref T). p x
										   then seq_app (\x2:(Ref T). b x) (f x)
									     else unit))) / st'
\end{lstlisting}


\subsection{Generator}
For my generator feature, there are four major terms, \textit{gen, gyield, giterate}, and \textit{gnext}. The \textit{gen} term is used to construct a iterable generator object. The \textit{gyield} term is used for trigger a \textit{exit} pair during the reduction of the \textit{seq} term. The \textit{giterate} term is invoked by the \textit{gnext} in order to update a iterable object's reference and return the next item of the iterable object.
\begin{itemize}
	\item 
	\textbf{ST_Gen3}: If $t_1$ is a generator function and $t_2$ is its corresponding input, then apply the function with the input to return a reference of a iterable object.
	\begin{lstlisting}
	Inductive gen_fun : tm -> Prop :=
	| Tgen_fun : forall x T x' t,
		gen_fun (abs x T 
								(tlet x' (ref (var x)) 
												 (ref (abs "_" Unit t)))).

	Reduction:

										t1 / st --> t1' / st'
							--------------------------------- (ST_Gen1)
								gen t1 t2 / st --> gen t1' t2 / st'

										t2 / st --> t2' / st'
							--------------------------------- (ST_Gen2)
							gen v1 t2 / st --> gen v1 t2' / st'
							
												gen_fun v1
							--------------------------------- (ST_Gen3)
							gen v1 v2 / st --> app v1 v2 / st
	\end{lstlisting}
\end{itemize}	
\begin{itemize}
	\item 
	\textbf{ST_Gyield}: A \textit{yield} term is a value if only if $t$ is a value.
	\begin{lstlisting}
	Reduction:

									  t / st --> t' / st'
							--------------------------------- (ST_Gyield)
						 	gyield t / st --> gyield t' / st'
	\end{lstlisting}
\end{itemize}
The first term of the \textit{giterate} term is a iterable object's reference. And the second one is the iterable object in the reference.
\begin{itemize}
	\item 
	\textbf{ST_GiterateExitPair1}: Evaluate the body $t$ of the generator.
	\item
	\textbf{ST_GiterateExitPair2}: If the body $t$ can be reduced to a \textit{exit} pair, then assign the rest sequences to the generator's reference and return the yield value.
	\item
	\textbf{ST_GiterateExitPair3}: If the body $t$ can be reduced to a \textit{yield} term, then assign unit to the generator's reference and return the yield value.
	\item
	\textbf{ST_GiterateExitPair4}: If the body $t$ can be reduced to a value, then assign a $unit$ to the generator's reference and return a $unit$.
	\begin{lstlisting}
	Reduction:

								t1 / st --> t1' / st'
			---------------------------------------  (ST_Giterate1)
			giterate t1 t2 / st --> giterate t1' t2 / st'

							
								t2 / st --> t2' / st'
			---------------------------------------  (ST_Giterate2)
			giterate v1 t2 / st --> giterate v1 t2' / st'


			-------------------------------  (ST_GiterateExitPair1)
			giterate (loc l) (abs "_" Unit t) / st
					--> giterate (loc l) t / st


			-------------------------------  (ST_GiterateExitPair2)
			giterate (loc l) (pair (yield v1) v2) / st
					--> [_:=assign (loc l) (ref v2)]v1 / st

							
			-------------------------------  (ST_GiterateExitPair3)
			giterate (loc l) (yield v1) / st
					--> [_:=assign (loc l) unit]v1 / st
			
			-------------------------------  (ST_GiterateExitPair4)
			giterate (loc l) v1 / st 
					--> [_:=assign (loc l) unit]unit / st
	\end{lstlisting}
\end{itemize}

The \textit{gnext} takes a iterable object's reference and returns the next item of the object.
\begin{itemize}
	\item 
	\textbf{ST_Gnext2}: If the $t$ is a reference term, then reduce to a \textit{giterate} term.
	\begin{lstlisting}
	Reduction:

								t / st --> t' / st'
					-------------------------------- (ST_Gnext1)
					gnext t / st --> gnext t' / st'

					
					-------------------------------- (ST_Gnext2)
					gnext (loc L) / st 
					--> giterate (loc L) !(loc L) / st'
	\end{lstlisting}
\end{itemize}

