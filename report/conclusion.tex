\section{Conclusion}
\label{sec:conclusion}
In this project, I extended lambda calculus with the generator feature to improve the memory efficiency of the iteration operation in Coq. The key idea to implement this feature is that, I defined a \textit{yield} term to trigger the \textit{exit} point and re-\textit{entry} point in the reduction of sequences in a function program. In this way, we can pause and resume a function program in order to iterate each generated item with constant space instead of storing all of them into a list. A generator feature demo is provided in both the Coq code file and below by comparing the ordinary function \textit{firstn} and the generator functionn \textit{gfirstn}. 
\begin{lstlisting}
firstn = \n:Nat.
         let num = ref 0 in 
         let nums = ref (nil Nat) in 
             while P B num;
             !nums

P = \num:(Ref Nat). !num < n
B = \num:(Ref Nat). 
    nums := !nums.append(!num); 
    num := succ(!num)
\end{lstlisting}
With the generator feature, we can not only utilize the iterative functionality in less code, but also access it through efficient memory management.
\begin{lstlisting}
gfirstn = \n:Nat.
          let num = ref 0 in 
              ref (\_:Unit (while P B num)

P = \num:(Ref Nat). !num < n
B = \num:(Ref Nat). 
    yield !num; 
    num := succ(!num)
\end{lstlisting}
Use \nt function to iterate each item of the generator.
\begin{lstlisting}
let g = gen gfirstn 10 in
let x = next(g) in
let x = next(g) in
    (x, y);
\end{lstlisting}
However, the generator cannot replace the list-based iteration due to the high cost in generating items. If we need to reuse the generated values multiple times, then the usage of a list is reasonable.

