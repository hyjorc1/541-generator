\section{Syntax}
\label{sec:approach}
Abstract syntax is a structure used in the representation of text in computer languages, which are stored in a tree structure as an abstract syntax tree. Here, the abstract syntax supporting for my language feature is provide below.
%
\begin{lstlisting}[basicstyle=\small]
Syntax:

			t ::=                		  Terms
				| ...										(other terms same as before)
				| seq t1 t2             sequence
				| ref t					   	    reference 
				| deref t               dereference
				| assign t1 t2          assignment
				| loc nat               reference cell location
				| let x=t1 in t2        let-binding
				| fix f   					    fixed-point operator
				| while p b             while function 
				| gen t1 t2             iterable generator
				| yeild t								yeild term
				| iterate t1 t2         iterate operator
				| next t                next operator

			T ::=                		  Types
				| ...               		
				| (Itr T)				        Iterator type
			
			v ::=                		  Values
				| ... 
				| loc l				          reference value
				| yield t               yield value
\end{lstlisting}
\begin{itemize}
	\item 
	\textbf{Terms}: The \textit{seq} term is used to execute each terms in a order. 
	The reference term is used to tracking variable value globally, so \textit{ref, deref, assign, loc} are needed. The \lt can bind a name \textit{x} to \textit{t1} while reduce \textit{t2}. 
	The \fix can be applied to a abstraction in order to return its \textit{fixed point} for recursion. 
	The \while takes a predicate function and a body function to form a while loop function. 
	The \gen takes a generator function and a argument used to pass into the generator. 
	The \textit{yield} term is used to trigger the exit point and re-entry point in a \textit{seq} term.
	The \textit{iterate} operator takes a iterable object's reference and the object as two arguments in oder to return the next item.
	The \nt term takes a iterable term.
	\item 
	\textbf{Types}: The $next$ operator should only take iterable terms. 
	So, I added a iterator type to identify the iterable generator. 
	\item 
	\textbf{Values}: Both \textit{loc l} and \textit{yield t} are value terms without any further reduction.
	\item 
	\textbf{Others}: For the other required terms, types, and values, I am not going to show more details here since they are similar to those terms in chapter \textit{MoreStlc}.
\end{itemize}