\section{Approach}

\subsection{Exit Point and Re-entry Point}
The key idea of the generator function is how to encounter and build exit point and re-entry point in the function. 
To solve this problem, the first step is to implement sequence and we can write 
\begin{lstlisting}[basicstyle=\small]
r; r := succ(r)
\end{lstlisting}
as an abbreviation for 
\begin{lstlisting}[basicstyle=\small]
(\x:Unit. (r := succ(r)) r
\end{lstlisting}
Let's define a \textit{yield} term in the sequence as shown below. Suppose this sequence is the body code of one function. If the function is called and the program flow encounters a \textit{yield} term, then the term would be regarded as an \textit{exit} point that can pause the program process to return the content in it and a re-entry point to resume the rest executions of the function.
\begin{lstlisting}[basicstyle=\small]
yield r; r := succ(r)
\end{lstlisting}
To achieve this goal, I designed a naive \textit{tseq} function shown as below that takes two terms \textit{t1} and \textit{t2}. If the \textit{t1} is a \textit{yield} term, then it returns an \textit{exit} pair of terms (t1', (abs "_" Unit t2)), where t1' is the actual content returned at that exit point and (abs "_" Unit t2) is the remainder executions of the function without further reduction.
\begin{lstlisting}[basicstyle=\small]
Definition tseq t1 t2 := 
match t1 with 
| yield t1' => (t1', (\_:Unit. t2))
| _ => (\_:Unit. t2) t1
end.
\end{lstlisting}
By applying our \textit{tseq} function, the sequence shown above would return a pair:
\begin{lstlisting}[basicstyle=\small]
(r,  (\_:Unit. r := succ(r))
\end{lstlisting}
However, I implemented the sequence as a \textit{seq} term with the similar structure to \textit{list} in my project instead of the \textit{tseq} function in order to ease sequence concatenation and nested sequence flattening.

\subsection{Yield in Iterations}
To make sure the \textit{seq} term would work in iterations, more explicitly the second term of the returned  \textit{exit} pair should contain the rest iterations of a loop, we need to define the loop recursively with \textit{fixed point}. A \while loop can be defined as below, where $p$ is a predicate function of input $r$.
\begin{lstlisting}[basicstyle=\small]
while (p(r)) {
	yield r; 
	r := succ(r)
}
\end{lstlisting}
Then we can define a \while function with the \textit{fixed point}.
\begin{lstlisting}[basicstyle=\small]
def while (r) {
	if p(x) do
		  yield r; 
	    r := succ(r);
		  while(r)
}
\end{lstlisting}
In my design, the \while function returns a \textit{seq} term without further reduction. Suppose a \while function is already one term in a body sequence of a function as shown below. Since my \while function returns a \textit{seq} term, we can flatten the nested body sequence into a one-dimensional sequence. In this way, a nested \textit{yield} term can still trigger a complete \textit{exit} pair $(yield$ $r,$ $t')$, where the $t'$ is a sequence of $r := succ(r); while(r); t3; t4$ containing the rest program of the function $f$. More details are provided in the operation semantic section.
\begin{lstlisting}[basicstyle=\small]
def f(r) {
	t1;
	t2;
	while(r);
	t3;
	t4
} 		
\end{lstlisting}

\subsection{Generator Feature}
I will show how my generator feature works in my language. First, let's give an example about an ordinary function \textit{F}. In line 2, the \textit{x} is the number argument passing into the function and \textit{x'} is the reference for \textit{x}. In line 3 and 4, we increase and decrease the number by 1 in the reference \textit{x'}. Then in line 5, we apply a \while term constructed by a predicate function $P$ and a body function $B$ to \textit{x'}. At last, we return the value of under the reference cell \textit{x'} in line 6.
\begin{lstlisting}[basicstyle=\small, numbers=left]
F = \x: Nat.
		let x' = (ref x) in 
		x' := succ (deref x');
		x' := pred (deref x');
		while P B x';
		!x'
\end{lstlisting}
The predicate and body functions in the \while term are: 
\begin{lstlisting}[basicstyle=\small]
P = \x':(Ref Nat). (deref x') < 5
B = \x':(Ref Nat). x' := succ !x'
\end{lstlisting}
Next, let's see an example for a generator function. The generator \textit{G} is similar to the function \textit{F}. However, the generator returns a reference of a unit function with a sequence body instead of a pure sequence term in two main reasons:
\begin{enumerate}
	\item A generator function should return a reference of a iterable object by given an input.
	\item A unit function can prevent the reduction process of the \textit{seq} term in the \textit{ref} term. Because I need the \textit{seq} term only be reduced in the \nt term which is used to iterate each items of a iterable object. 
\end{enumerate}
\begin{lstlisting}[basicstyle=\small, numbers=left]
G = \x: Nat.
		let x' = (ref x) in 
		ref ( \_: Unit.
					x' := succ (deref x');
					yield !x';
					x' := pred (deref x');
					yield !x';
					while P B x';
					yield !x')
\end{lstlisting}
The predicate and body functions in the \while term are:
\begin{lstlisting}[basicstyle=\small, numbers=left]
P = \x':(Ref Nat). (deref x') < 5
B = \x':(Ref Nat). 
		yield !x'; 
		x' := succ !x'
\end{lstlisting}
To make this work, I defined a \gen term to construct a iterable generator object in type $Itr$ $T$ by taking a generator function and a corresponding input. 
I also implemented a \nt function shown as below in order to iterate each item returned at the \textit{exit} point of a \gen term. 
Here, the $x, y$ are the first two items returned from the generator $g$.
\begin{lstlisting}[basicstyle=\small, numbers=left]
let g = gen G x in 
let x = next(g) in
let y = next(g) in
(x, y)
\end{lstlisting}
In the following sections, I provide the design of my implementation including syntax, operation semantics, and typing of this generator feature. 